
\documentclass[a4paper,12pt]{article}
\usepackage[utf8]{inputenc}
\usepackage[ngerman]{babel}
\usepackage{geometry}
\usepackage{enumitem}
\geometry{a4paper, margin=2.5cm}

\title{Interviewleitfaden: Künstliche Intelligenz im Studium}
\author{Irmak Damla Özdemir}
\date{17. Dezember 2024}

\begin{document}

\maketitle

\section*{Interviewleitfaden}

\textbf{Thema:} Küntsliche Intelligenz Im Studium

\begin{enumerate}
\item \textbf{Einleitung}
\begin{enumerate}[label=\alph*)]
\item Interviewer stellt sich vor
\item Interviewer stellt Projekt vor
\item Interviewer informiert über Dauer und Ablauf des Interviews
\end{enumerate}

\item \textbf{Vorstellung des Teilnehmers}

\item \textbf{Künstliche Intelligenz im Studium}
\begin{enumerate}[label=\alph*)]
    \item \textbf{Einleitende Sätze} \\
   Unser Ziel ist es, ein tiefergehendes Verständnis für die gegenwärtigen Herausforderungen und potenziellen Lösungsansätze im Studium mit KI zu erlangen. Dafür haben wir eine Reihe von Fragen vorbereitet. \\
    
    \item \textbf{Fragen}
    \begin{enumerate}[label=\roman*)]
    \item In welchen Bereichen finden Sie KI-Anwendungen hilfreich? Welche Erfahrungen haben Sie darüber gesammelt? 
        \item  In welchen Situationen nutzen Sie KI am häufigsten?
        \item Welche spezifischen KI-Tools oder -Anwendungen nutzen Sie regelmäßig in
Ihrem Studium?
        
    \end{enumerate}
\end{enumerate}

\item \textbf{Problem}
\begin{enumerate}[label=\alph*)]
    \item \textbf{Einleitender Satz} \\
    Sie haben schon ein paar Probleme erwähnt, die bei der Nutzung von Künstliche Intelligenz im Studium auftreten. Wir würden jetzt gerne näher auf die Probleme eingehen.
    
    \item \textbf{Fragen}
    \begin{enumerate}[label=\roman*)]
        \item  Gibt es Situationen, in denen KI sich negativ auf Sie ausgewirkt hat? Können
        Sie dazu ein Bespiel nennen?

        \item Haben Sie Bedenken hinsichtlich der Datenschutz- oder Sicherheitsaspekte bei
der Nutzung von KI im Studium?

\item  Haben Sie weitere Bedenken?



    \end{enumerate}
\end{enumerate}

\item \textbf{Verbesserungen und Erwartungen}
\begin{enumerate}[label=\alph*)]
    \item \textbf{Einleitender Satz} \\
    Das Ziel dieses Projekts ist es, die Nutzung von Künstliche Intelligenz im Studium zu verbessern. In dem letzten Teil würden wir gerne auf Verbesserungsmöglichkeiten eingehen.
    
    \item \textbf{Fragen}
    \begin{enumerate}[label=\roman*)]
        \item Sollte die Nutzung von KI im Studium völlig frei gestaltet sein oder sollten
Grenzen gesetzt werden?
        \item Welche Erwartungen haben Sie an zukunftige KI-gestützte Lernangebote? 

        \item Weitere Bedenken?


    \end{enumerate}
\end{enumerate}

\item \textbf{Abschluss des Interviews}
\begin{enumerate}[label=\alph*)]
    \item Bedanken
    \item Folgenden Kontakt besprechen
\end{enumerate}

\end{enumerate}


\newpage
\section*{Interviewprotokoll}

\textbf{Interviewdatum:} 17. Dezember 2024 \\
\textbf{Interviewpartner/in:} Buse Okcu \\
\textbf{Sitzordnung:} Ecke \\
\textbf{Ort:} Technische Hochschule Würzburg-Schweinfurt, Sanderheinrichsleitenweg 20, Würzburg, Deutschland  \\
\textbf{Dauer:} 15 Minuten

\subsection*{Zusammenfassung der Interviewergebnisse}
\begin{enumerate}[label=\arabic*.)]
\item \textbf{Erfahrungen mit KI im Studium:} 
    \begin{enumerate}[label=\roman*)]
        \item Ich finde KI-Anwendungen besonders hilfreich in der Literaturrecherche
        und beim Verfassen von Texten, da sie komplexe Inhalte zugänglicher
        machen und den Lernprozess effizienter gestalten. In meinem Studium
        habe ich die Erfahrung gemacht, dass KI nicht nur Zeit spart, sondern
        auch dabei hilft, kreative Lösungen für Aufgaben zu finden.

        \item Am häufigsten nutze ich KI, wenn ich schwierige Themen analysieren oder
        große Mengen an Informationen strukturieren muss. Besonders vor
        Abgabeterminen oder bei der Vorbereitung auf Prüfungen ist sie ein
        unverzichtbares Hilfsmittel.

        \item Zu den spezifischen Tools, die ich regelmäßig nutze, gehören ChatGPT für
        thematische Analysen und DeepL für Übersetzungen. Diese Anwendungen
        bieten nicht nur praktische Unterstützung, sondern fördern auch meine
        Eigenständigkeit im Studium.
    \end{enumerate}

    
    \item \textbf{Herausforderungen:}
    \begin{enumerate}[label=\roman*)]
        \item Es gab eine Situation, bei der ich KI für eine
        wissenschaftliche Arbeit genutzt habe, um Informationen zu
        recherchieren. Anfangs dachte ich, die Daten wären korrekt, aber später
        stellte sich heraus, dass einige Informationen unvollständig oder sogar
        falsch waren. Das hat dazu geführt, dass ich mehr Zeit für die
        Überprüfung und Korrektur aufwenden musste. Solche Erfahrungen
        zeigen mir, dass man KI-Ergebnisse immer kritisch hinterfragen sollte und
        nicht einfach blind vertrauen kann.

        \item Ja, ich habe durchaus Bedenken, insbesondere wenn es um die
        Verarbeitung persönlicher oder sensibler Daten geht. Viele KI-Tools
        speichern Nutzerdaten, und es ist oft unklar, wie diese verwendet werden.
        Als Studentin achte ich darauf, keine vertraulichen Informationen in solche
        Tools einzugeben und verwende sie nur in einem Rahmen, der
        datenschutzkonform ist.
    \end{enumerate}
    
    \item \textbf{Verbesserungen und Erwartungen:}
    \begin{enumerate}[label=\roman*)]
        \item Meiner Meinung nach sollte die Nutzung von KI im Studium nicht völlig frei
        gestaltet sein. Aber es ist wichtig, gewisse Grenzen zu setzen, zum
        Beispiel im Bereich Datenschutz oder um sicherzustellen, dass
        Studierende nicht zu abhängig von KI werden. Gleichzeitig finde ich, dass
        diese Grenzen nicht zu starr sein sollten, damit wir KI weiterhin kreativ und
        effektiv nutzen können. Eine klare Orientierung, wie man KI
        verantwortungsvoll einsetzt, wäre hier hilfreich.

        \item Ich erwarte, dass KI-gestützte Lernangebote in Zukunft stärker
        personalisiert werden. Es wäre hilfreich, wenn sie Schwächen erkennen
        und darauf basierend spezielle Übungen oder Lernpläne vorschlagen
        könnten. Außerdem ist es mir wichtig, dass die Datennutzung
        transparenter wird, damit man diese Tools sicher verwenden kann.
        Funktionen, die die Zusammenarbeit in Gruppen unterstützen, wären
        ebenfalls sehr nützlich.
    \end{enumerate}
\end{enumerate}

\newpage
\subsection*{Reflektion des Interviews Als Interviewerin}
\textbf{Erkenntnisse/Erfahrungen:} \\\\
- Das Gespräch verlief flüssig, und jede Frage wurde beantwortet, ohne dass etwas offen blieb. \\
- Durch das Interview konnte ich aus der Perspektive einer Studentin besser verstehen, wie KI im Studium genutzt wird und in welchen Bereichen sie besonders hilfreich ist.\\\\
\textbf{Herausforderungen:}\\\\
- Die Interviewte war bei ihrem ersten Interview deutlich nervös, und es hat einige Zeit gedauert, diese Nervosität zu überwinden.\\
- Die Antworten der Interviewten waren anfangs recht oberflächlich, was mich darüber nachdenken ließ, ob die Fragen noch detaillierter formuliert werden könnten, um tiefergehende Ergebnisse zu erzielen.\\\\
\textbf{Positive Aspekte:}\\\\
- Die Interviewte zeigte großes Interesse und Neugier für das Interview. Ihr Informatikstudium half ihr dabei, ein breites Wissen über verschiedene KI-Tools einzubringen.\\
- Der Leitfaden erwies sich als sehr hilfreich, da er das Gespräch strukturiert und übersichtlich gehalten hat.

\subsection*{Reflektion des Interviews Als Interviewten}
\textbf{Erkenntnisse/Erfahrungen:} \\\\
- Die Interviewfragen waren interessant und gut durchdacht. Das Interview verlief strukturiert, und der Fokus blieb stets auf dem Thema.\\
- Die Gestik und Mimik des Interviewers trugen zu einer entspannten Atmosphäre bei, und seine freundliche Körpersprache unterstützte den gesamten Interviewprozess positiv.\\\\
\textbf{Herausforderungen:}\\\\
- Es hätten innovativere und kreativere Fragen gestellt werden können, um das Interesse am Thema noch stärker zu wecken.\\
- Einige Fragen könnten tiefergehend formuliert sein, um das Gespräch noch dynamischer und spannender zu gestalten.\\\\
\textbf{Positive Aspekte:}\\\\
- Der Interviewer erklärte die Fragen ausführlich und fragte aktiv nach, ob ich noch etwas hinzufügen möchte. Dies hinterließ einen positiven Eindruck.\\
- Die Offenheit und Geduld des Interviewers sorgten dafür, dass ich mich sicher und wertgeschätzt fühlte, was meine Antworten positiv beeinflusste.



\subsection*{Verbesserungsvorschläge für zukünftige Interviews (Nachfolgende Reflektion)}
\subsection*{Als Interviewerin}
\textbf{Bereits addressierte Aspekte}} \\\\
- Ein klarer Ablauf wurde geschaffen, und die Antworten der Interviewten wurden gleichzeitig protokolliert.\\\\
\textbf{Nicht addressierte Aspekte und Optimizierung}\\\\
- Der Leitfaden bzw. der Ablauf des Interviews wurde der Interviewten nicht ausführlich genug erklärt, was möglicherweise Unsicherheiten verursacht hat.\\
- Die Sitzordnung war gegenüberliegend. Stattdessen hätte eine Ecke als Sitzordnung gewählt werden können, um die Atmosphäre entspannter zu gestalten und die Effizienz des Interviews zu erhöhen.

\subsection*{Als Interviewten}
\textbf{Bereits addressierte Aspekte}} \\\\
- Die Interviewstruktur war klar definiert, was mir half, mich auf die Fragen zu konzentrieren und strukturierte Antworten zu geben. Dies entspricht der Best Practice, dass ein Leitfaden mit klaren Themenblöcken die Durchführung erleichtert.\\\\
\textbf{Nicht addressierte Aspekte und Optimizierung}\\\\
- Es hätte vorab eine Erklärung geben können, dass es keine „richtigen“ oder „falschen“ Antworten gibt, um die Nervosität noch weiter zu verringern. Solche Maßnahmen fördern Offenheit und Ehrlichkeit.\\\\\\


\noindent\textbf{Unterschrift Interviewer/in:} \underline{\hspace{6cm}} \\
\textbf{Unterschrift Interviewpartner/in:} \underline{\hspace{6cm}}

\end{document}